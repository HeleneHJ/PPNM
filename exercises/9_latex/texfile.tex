\documentclass[a4paper,twocolumn,11pt]{article}
\usepackage[utf8]{inputenc}
\usepackage[english]{babel}
\usepackage{graphicx}
\usepackage{biblatex}
\usepackage{cleveref}
\title{Using \LaTeX{} and Gnuplot: The Fresnel Integrals}
\author{Helene Hestbech Jørgensen}
\date{\today}
\begin{document}
\maketitle

\noindent
The Fresnel integrals are defined as the following definite integrals:
\begin{equation}\label{fres-sine}
	S(x)=\int_{0}^{x} \sin\left(t^2\right) dt,
\end{equation}
and
\begin{equation}\label{fres-cos}
	C(x)=\int_{0}^{x} \cos\left(t^2\right) dt,
\end{equation}

\noindent where $S(x)$ is called the Fresnel sine integral and $C(x)$ is called the Fresnel cosine integral. These functions are shown in the figure.

They are named after the French engineer and physicist, Augustin-Jean Fresnel (1788-1827), who originally used them in optics calculations.\

The limit of the Fresnel functions as $x$ approaches infinity are

\begin{equation}
\int_{0}^{\infty} \cos t^2 dt = \int_{0}^{\infty} \sin t^2 dt = \frac{\sqrt{2\pi}}{4} \approx 0.6267.
\end{equation}

%The maximum of $C(x)$ is about $0.977451424$. 

In some cases, the argument $t^2$ is replaced with $\pi t^2/2$ and the functions are known as normalized Fresnel integrals. These functions converge to 1/2.\\


The Fresnel integrals can be expanded in the following power series that converge for all x:
\begin{equation}
	S(x)=\sum_{n=0}^{\infty} (-1)^n \frac{x^{4n+3}}{(2n+1)!(4n+3)},
\end{equation}
and 
\begin{equation}
C(x)=\sum_{n=0}^{\infty} (-1)^n \frac{x^{4n+1}}{(2n)!(4n+1)}.
\end{equation}

The Fresnel integrals can be extended to the domain of complex numbers, $S(z)$ and $C(z)$, where they become analytic functions of a complex variable $z$ and can be expressed using the error function.

\begin{figure}[h]\label{fresnel}
\input{Fresnel.tex}
\caption{Plots of the Fresnel functions, $S(x)$ and $C(x)$.}
\end{figure}

\begin{thebibliography}{9}
	\bibitem{Wikifresnel}
	Wikipedia, Fresnel integral,
	\url{https://en.wikipedia.org/wiki/Fresnel_integral},
	accessed: 2020-05-05.
	
	\bibitem{WikiJean}
	Wikipedia, Augustin-Jean Fresnel,
	\url{https://en.wikipedia.org/wiki/Augustin-Jean_Fresnel},
	accessed: 2020-05-05.
\end{thebibliography}

\end{document}
